
\section{Discussion}

In this paper we have presented techniques that can speed up the
bottleneck convolution operations in the first layers of a convolutional network by a factor $1.6-2$, with a
moderate loss of performance ($\sim1\%$). The empirical speedups
achieved are still some way short of the theoretical gains, thus
further improvements might be expected with further engineering effort. 
Moreover, the techniques are orthogonal
to other approaches for efficient evaluation, such as quantization or
working in the Fourier domain. Hence, they can potentially be used
together to obtain further gains. 

Given the widespread use of neural networks in large scale industrial
settings, the speed gains we demonstrate have significant economic
value. Companies such as Google or Facebook process $10^8$ images/day,
requiring many thousands of machines. Our speed gains mean that the
number of machines dedicated to running convolutional networks could be roughly halved, so saving millions of
dollars, as well as a significant reduction in environmental impact.


We also show that our methods reduce the memory footprint of weights
in the first two layers by factor of $2-3\times$. When applied to the whole network,
this would translate into a significant savings, which would facilitate mobile deployment
of convolutional networks.

Our approximations are applied to fully trained networks. The
small performance drops that result could well be mitigated by further
training the network after applying the approximation. By alternating
these two steps, we could potentially achieve further gains in
speed-up for a modest performance drop.


Another aspect of our technique is regularization. It seems that
approximated filters look cleaner, and that sporadically we get better
test error (e.g.~bottom left of Figure
\ref{fig:biclustering_theory}(left)). We would like to experiment with
low-rank projections during training as a regularization
technique. Effectively it decreases number of learnable parameters,
so it might improve generalization, a major issue with large convolutional networks.
  


% Presented methods were applied to fully trained networks. Obviously, networks were not trained
% with constrains on connectivity or sparsity as are present in approximated network. This means 
% that agresive approximations, which can have much larger speedup are likely to give worse error score. 
% Training with new connectivity or sparsity constraint might be quite difficult. Constrains 
% comming from approximation would have to be included in optimization, and a lot of additional coding would be
% neccessary. We propose different strategy. 


% It is easy to decompose weights using any of our method, and then reconstruct them. Particularly,
% reconstructed weights can be approximated exactly with the same decomposition. Let's call this process
% projection. Proposed strategy is to take a fully trained network, and iterativelly repeat (1) apply projection, 
% (2) train for few epochs. Hopefuly, this way weights will become expressable in desired way, which can be speededup
% more than $2$ times, and prediction score would be preserved due to fine-tunning. From engineering perspective
% stage (1) can be implemented in any external easy environment (e.g. MATLAB), while stage can stay as it is. 
% Stage (1) is executed every few epochs, so it is not a computational bottle-neck.



% and thus our speed gains 

% the speed gains we demonstrate have 
% Speeding up the evaluation of neural networks has a high industrial impact. Computation
% of forward pass on all public internet available images (billions of images), or
% on all social media data cost in milions of dollars for a single pass. High costs
% of evaluation limits use of neural networks, and might forces companies to rely
% on inferior object recognition techniques which are much cheaper in evaluation 
% (e.g. SIFT features). 



% Presented in this paper techniques show how to achieve speedup by the factor of $2$,
% with moderate loss of performance ($< 1\%$). Moreover, they are ortogonal 
% to previously developed techniques like quantization, or working in Fourier domain, and
% can be used together to further boost evaluation speed. Finally, we believe that fine-tunning
% technique described in Section \ref{sec:fine_tunning} can enable even further 
% agresive approximations, and give a higher boost.



%\subsection{Fine-tunning}\label{sec:fine_tunning}

Presented methods were applied to fully trained networks. Obviously, networks were not trained
with constrains on connectivity or sparsity as are present in approximated network. This means 
that agresive approximations, which can have much larger speedup are likely to give worse error score. 
Training with new connectivity or sparsity constraint might be quite difficult. Constrains 
comming from approximation would have to be included in optimization, and a lot of additional coding would be
neccessary. We propose different strategy. 


It is easy to decompose weights using any of our method, and then reconstruct them. Particularly,
reconstructed weights can be approximated exactly with the same decomposition. Let's call this process
projection. Proposed strategy is to take a fully trained network, and iterativelly repeat (1) apply projection, 
(2) train for few epochs. Hopefuly, this way weights will become expressable in desired way, which can be speededup
more than $2$ times, and prediction score would be preserved due to fine-tunning. From engineering perspective
stage (1) can be implemented in any external easy environment (e.g. MATLAB), while stage can stay as it is. 
Stage (1) is executed every few epochs, so it is not a computational bottle-neck.




